%https://www.overleaf.com/project/5f65e03fb51c4600017589e3

\documentclass{article}
\usepackage[utf8]{inputenc}

\title{tomachello}
\author{tomachello3@gmail.com}
\date{}

\usepackage{geometry}
 \geometry{
 a4paper,
 total={170mm,257mm},
 left=20mm,
 top=20mm,
 }

\usepackage{amsmath, amssymb}

\newcommand{\st}{\textbf{ s.t. }}
\newcommand{\SP}{\text{ }}
\newcommand{\problem}[1]{\subsubsection*{\textbf{problem \##1}}}
\newcommand{\chapter}[1]{\section*{#1}}

\newcommand{\Z}{\mathbb{Z}}
\newcommand{\R}{\mathbb{R}}
\newcommand{\F}{\mathbb{F}}
\newcommand{\Zpf}{\mathbb{Z}_{\ge0}}
\newcommand{\Zpv}{\mathbb{Z}_{\ge1}}
\newcommand{\Rpf}{\mathbb{R}_{\ge0}}
\newcommand{\Rpv}{\mathbb{R}_{>0}}

\newcommand{\CST}[1]{\textsc{#1}}
\newcommand{\FNC}[1]{\texttt{#1}}
\newcommand{\VAR}[1]{\textit{#1}}

\begin{document}
\maketitle

\subsection*{introduction to the tomachello project}
goal: creating a comprehensive mathematics book, pedagogically suited for gifted youth.\\
motivation: to write the book I needed growing up.\\
list of guidelines:\\
1. the reader has the job of discovering the math, while my job is to choose the exercises, guide the discussion, simplify and clarify the ideas and technique, and most importantly challenge the reader and make them love math.\\
2. in presenting an argument I must focus on the key idea, taking all details for granted - mathematical exposition should not be confusing to follow.\\
3. if a reader feels I did not follow one of the above guidelines they should let me know.\\
current state of project: collecting problems

\chapter{words}
\problem{bvo} if two words commute, they must both be powers of a third word. 
$$\{(w_1,w_2)\in \CST{words}^2 \st w_1w_2=w_2w_1\}=\{(w^\alpha,w^\beta)\st w\in\CST{words},\SP\alpha,\beta\in\Zpf\}$$
\problem{att} the set of all periods of a given word is closed under gcd, generally speaking.
$$\alpha,\beta\in \FNC{periods}(\VAR{word}), \SP\alpha+\beta-\FNC{gcd}(\alpha,\beta)\le \FNC{length}(\VAR{word})\implies \FNC{gcd}(\alpha,\beta)\in\FNC{periods}(\VAR{word})$$
\problem{fll} suppose you found two equal consecutive sub words of some \VAR{word} of length $\FNC{minimal period}(\VAR{word})-1$. then they differ in placement by a period of \VAR{word}.
\chapter{algorithms}
\problem{wcd} consider the following lock. you have an $n\times n$ table, in each square is a 
(decimal, 0-9) digit. to open the lock you need all digits to be zero. now, for each row and for each column there is a button allowing you to increase each digit in said row/column by one. (so that 9 goes to 0). but that's it, you cannot operate on a single square (digit). now i mess up the lock (starting from an open state and pushing lots of buttons randomly). how can you open it in $9n$ steps?
\problem{xbp} given a source of random bits - i.e. given $\FNC{random bit}=\FNC{random uniform}(\{1,2\})$, construct $\FNC{random uniform}(\{1,\dots,n\})$ using $\log_2(n)+O(1)$ average cost
\problem{upl} generate random permutation using random number generator
\problem{ocj} longest common sub sequence of two words
\problem{uhv} finding largest $j-i$ where $i<j$ and $a_i>a_j$ in a long sequence
\chapter{information}
\problem{see} alice tells bob $n$, as well as the elements of a sphere $S=S(x)$ in the hamming cube $Q_n$. how should bob find $x$?
\problem{zgj} thirteen objects are displayed to the audience. they choose two, and tell them to bob. then alice walks in the room. bob now tells alice that a certain bulb is not one of the two chosen ones. now alice gets four attempts at finding the two chosen ones out of the twelve. how is the trick done?
\problem{aqd} on each of the 64 squares of a chessboard is a coin- some heads, some tails. the audience chooses a certain square on the board and tell bob. bob then flips one coin, after which alice comes in the room, looks only at the chessboard, and after some thought reveals the audiences square. how is the trick done?
\problem{qoj} there's $n$ people in a row, and everyone has a hat with a different number from $1$ to $n+1$. everyone sees only those in front of them. from the end of the line, they need to guess their number, hearing the answers of the people before them. they cannot guess an answer that was already said. how can they maximize the number of correct answers?
\problem{yfg} there's $n$ people in a row, and everyone has a hat with a number from $1$ to $k$ (start with $k=2$). everyone sees only those in front of them. from the end of the line, they need to guess their number, hearing the answers of the people before them. how can they maximize the number of correct answers?
\problem{uzb} there's $n$ people in a circle, and everyone has a hat with a number from $1$ to $n$. (start with $n=2$) everyone sees everyone but themselves. they all simultaneously guess their color. how can they guarantee that at least one person gets it right?
\problem{hrm} !!haven't solved yet!! there's $9^n$ coins, all weighing the same except for one which is lighter. you have three scales (you can put some coins on one side and some on the other and it'll tell you which side weighs more) but one scale is broken and can output randomly/adversely. how to find lighter coin in $3n+1$ steps?
\problem{tkg} !!haven't solved yet!! Alice has a map of Wonderland, a country consisting of n towns. For every
pair of towns, there is a narrow road going from one town to the other. One day, all the roads
are declared to be “one way” only. Alice has no information on the direction of the roads, but
the King of Hearts has offered to help her. She is allowed to ask him a number of questions.
For each question in turn, Alice chooses a pair of towns and the King of Hearts tells her the
direction of the road connecting those two towns.
Alice wants to know whether there is at least one town in Wonderland with at most one
outgoing road. Prove that she can always find out by asking at most 4n questions.
\problem{yuq} there's $n$ people, some good and some bad. you don't know which is which, only that there are more good than bad. you can ask any if any other is good or bad, and they'll tell you. if they are bad, they can lie. how can you find out which is which in, say, $2n-2$ questions?\\\\
i wonder, if we know there's at least, say $80\%$ good people, what's the cost now, and how should we do it?
\chapter{invariants}
\problem{oob} there is a circular bracelet with a red diamond, blue diamond, and a green diamond. (in this order, clockwise). now, you're allowed to take two consecutive diamonds with the same color, and put in between them a diamond of any color. you're also allowed to take two consecutive diamonds with a different color, and put in between them a diamond of one of those two colors. and you're allowed to perform those operations in reverse. if you see three consecutive diamonds not consisting of all three colors, you may delete the middle one. show you cannot get to red, green, blue clockwise.
\chapter{diophantine}
\problem{zoj} $x^2+x=y^4+y^3+y^2+y+1$
\chapter{inequalities}
\problem{xbc} suppose $a_1,\dots,a_{2m+1}$ is positive geometric. then the average of the odd index elements is at least that of the even index elements.
\chapter{probability}
\problem{ifb} monty hall
\chapter{combinatorics}
\problem{lyr} at a certain mathematical conference, every pair of mathematicians are either friends or strangers.
At mealtime, every participant eats in one of two large dining rooms. Each mathematician insists upon eating in a room which contains an even number of his or her friends. Prove that the number of ways that the mathematicians may be split between the two rooms is a power.
\problem{aoz} !!haven't solved yet!! On a flat plane in Camelot, King Arthur builds a labyrinth L consisting of n walls,
each of which is an infinite straight line. No two walls are parallel, and no three walls have a
common point. Merlin then paints one side of each wall entirely red and the other side entirely
blue.
At the intersection of two walls there are four corners: two diagonally opposite corners
where a red side and a blue side meet, one corner where two red sides meet, and one corner
where two blue sides meet. At each such intersection, there is a two-way door connecting the
two diagonally opposite corners at which sides of different colours meet.
After Merlin paints the walls, Morgana then places some knights in the labyrinth. The
knights can walk through doors, but cannot walk through walls.
Let k(L) be the largest number k such that, no matter how Merlin paints the labyrinth L,
Morgana can always place at least k knights such that no two of them can ever meet. For
each n, what are all possible values for k(L), where L is a labyrinth with n walls?
\problem{iso} Let $n$ be a positive integer. Harry has $n$ coins lined up on his desk, each showing
heads or tails. He repeatedly does the following operation: if there are $k>0$ coins showing heads, then he flips the $k^\text{th}$ coin over; otherwise he stops the process. (For example, the
process starting with THT would be THT, HHT, HTT, TTT, which takes three steps.)
Letting $C$ denote the initial configuration (a sequence of $n$ H’s and T’s), write len(C) for the
number of steps needed before all coins show T. Show that this number len(C) is finite, and
determine its average value over all $2^n$ possible initial configurations C.
\problem{aqj} !!haven't solved yet!! In the nation of Onewaynia, certain pairs of cities are connected by roads. Every road connects exactly two cities (roads are allowed to cross each other, e.g., via bridges). Some roads have a traffic capacity of 1 unit and other roads have a traffic capacity of 2 units. However, on every road, traffic is only allowed to travel in one direction. It is known that for every city, the sum of the capacities of the roads connected to it is always odd. The transportation minister needs to assign a direction to every road. Prove that he can do it in such a way that for every city, the difference between the sum of the capacities of roads entering the city and the sum of the capacities of roads leaving the city is always exactly one.
\chapter{hypercube}
\problem{isj} suppose $f:Q_n\to \{\pm1\}$ is almost a homomorphism, in the sense that $f(x+y)\neq f(x)f(y)$ in probability at most $\varepsilon$. then there is a homomorphism $g$ such that $f\neq g$ in probability at most $\varepsilon$.
\chapter{analysis}
\problem{psq} suppose $f:\Zpv^2\to\Zpv$ is such that every positive integer is an image of at most $N$ points, for some fixed $N$. then $\displaystyle{\sup_{n,m}} \dfrac{f(n,m)}{nm}=\infty$
\chapter{processes}
\problem{kep} there's $n$ rooms in a circular palace. in each room lies a dragon or a princess. $n$ knights came into the palace. on his arrival, each knight received a certain room (so that no two knights receive the same room - every room got one knight) and went to that room. if a knight sees a dragon, he slays it, and turns it into a princess, and then keeps going to the next room. if he sees a princess, he kisses her, making her into a dragon which kills him. now, there may be many knights around the palace, but there can never be two in the same room simultaneously. what happens eventually?
\chapter{Euclidean geometry}
\problem{fou} two circles intersect at $A$. two points, starting simultaneously from $A$, trace the circles, both in the same orientation and constant angular speed [they meet at $A$ again after each revolved once around its
circle] prove the existence of a point $P$, such that the two points are always equidistant from $P$.
\problem{xpo} two circles of equal radii are tangent to a larger circle from the inside, at points $A,B$. let $M$ be some point on the larger circle, and let $A',B'$ be the intersections of $MA,MB$ with the corresponding small circle. then $AB||A'B'$
\chapter{harmonic functions}
\problem{bjy} a function $f:\Z^d\to[0,1]$ such that the average of $f$ around each point (on its $2d$ neighbors, two in each axial direction) equals the value of $f$ at that point (i.e. bounded harmonic function on lattice) must be constant.
\chapter{graphs}
\problem{cju} color the complete graph's edges red and blue. show that for one of those colors, the diameter of the graph is at most three. (given any two cities, you can reach one from the other using at most two other cities along the way)
\chapter{number theory}
\problem{rfh} determine $\displaystyle{\prod_{\alpha\in\F_p}(\alpha^2+1)}$\\\\
(don't remember the exact details but i'd be surprised if one can't extend the argument to $\F_q$)
\problem{euf} determine $\FNC{det}((\FNC{gcd}(i,j))_{1\le i,j\le n})$
\problem{ujm} fix $n,m$. what's the minimal degree of a polynomial $f$ such that $f(t)=\FNC{fibonacci}(t)$ for all $t=m,m+1,\dots,m+n$
\chapter{discrete geometry}
\problem{rbu} Theseus starts at the plane origin. He can go one unit in each of the four directions. After a sequence
of more than two such moves, starting with a step one unit to the south, he finds himself back at the origin. He never visited any point other than the origin more than once, and only visited the origin at the start and end of this sequence of moves. Let $X$ be the number of times that Theseus took a step one unit to the north, and then a step one unit to the west immediately afterward. Let $Y$ be the number of times that Theseus took a step one unit to the west, and then a step one unit to the north immediately afterward. Prove that $|X-Y| = 1$
\problem{iwg} given a plane triangle whose vertices are lattice points (lattice is $\Z^2$) but whose perimeter contains no other lattice points. inside the triangle are four more lattice points. then these four points are collinear.
\chapter{infinite combinatorics}
\problem{rut} any infinite real sequence has an infinite monotonic sub sequence.
\problem{obr} Ramsey's theorem
\end{document}

\subsection*{notation and other matters you should never think about}
variable quantities are denoted with \VAR{this text}, (example:\VAR{x,y,element,functional}). functions (of variable quantities) are denoted with \FNC{this text} (example:\FNC{gcd,power-set}) being a rigid map without a choice in the matter, given the variable has been set. constant quantities (or entities) are denoted by \CST{this text}, for they are fixed. example: $\CST{pi, fibonacci-word, quaternion-group}$. whenever there is a discussion on a certain object that can be any element of a set - i.e. a variable, but we fix them, we treat them as constants. for example, instead of writing\\ $\VAR{words}=\FNC{concatenation-span}(\VAR{alphabet})$, we have\\ $\CST{words}=\FNC{concatenation-span}(\CST{alphabet})$
