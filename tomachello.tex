\documentclass{article}
\usepackage[utf8]{inputenc}

\title{tomachello}
\author{tomachello3@gmail.com}
\date{}

\usepackage{amsmath, amssymb}


\newcommand{\st}{\textbf{ s.t. }}
\newcommand{\SP}{\text{ }}
\newcommand{\problem}[1]{\subsubsection*{problem \# #1}}
\newcommand{\chapter}[1]{\section*{#1}}

\newcommand{\Zpf}{\mathbb{Z}_{\ge0}}
\newcommand{\Zpv}{\mathbb{Z}_{\ge1}}
\newcommand{\Rpf}{\mathbb{R}_{\ge0}}
\newcommand{\Rpv}{\mathbb{R}_{>0}}

\newcommand{\set}[1]{\textsc{#1}}

\begin{document}
\maketitle

\subsection*{introduction to the tomachello project}
goal: creating a comprehensive mathematics book, pedagogically suited for gifted youth.\\
motivation: to write the book I needed growing up.\\
list of guidelines:\\
1. the reader has the job of discovering the math, while my job is to choose the exercises, guide the discussion, simplify and clarify the ideas and technique, and most importantly challenge the reader and make them love math.\\
2. in presenting an argument I must focus on the key idea, taking all details for granted - mathematical exposition should not be confusing to follow.\\
3. if a reader feels I did not follow one of the above guidelines they should let me know.

\chapter{words}
\problem{aaa} if two words commute, they must both be powers of a third word. 
$$\{(w_1,w_2)\in \set{words}^2 \st w_1w_2=w_2w_1\}=\{(w^\alpha,w^\beta)\st w\in\set{words},\SP\alpha,\beta\in\Zpf\}$$
\end{document}
